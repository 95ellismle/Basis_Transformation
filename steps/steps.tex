\documentclass[12pt,a4paper]{article}
\usepackage[utf8]{inputenc}
\usepackage[english]{babel}
\usepackage{amsmath}
\usepackage{amsfonts}
\usepackage{amssymb}
\usepackage{breqn}
\usepackage{graphicx}
\begin{document}Original Equation was:

\begin{dmath*}
\sum\limits_{k,z,n,m} \left[ U_{m,z} U_{n,z} u_{n}+ u_{k}\right]
\end{dmath*}

Split summations with a + or - in (this is a code thing rather than a maths thing):

\begin{dmath*}
\sum\limits_{k,z,n,m}  U_{m,z} U_{n,z} u_{n}+
\sum\limits_{k,z,n,m}   u_{k} 
\end{dmath*}

Simplifying the U terms using the relationship $\sum\limits_{x}U^{*}_{bx}U_{ax} = \delta_{ab}$:

\begin{dmath*}
\sum\limits_{k,z,n,m}   u_{n} \delta_{m,n}+
\sum\limits_{k,z,n,m}   u_{k} 
\end{dmath*}

Let them Kronecker deltas work their magic: \begin{dmath*}
\sum\limits_{k,n}   u_{n} +
\sum\limits_{k,z,n,m}   u_{k} 
\end{dmath*}